\documentclass[UTF8,12pt]{article}
\usepackage{CTEX}
\usepackage{graphicx}
\usepackage{geometry}
%opening
\title{线段求交问题}
\author{吴天}

\begin{document}
\maketitle
\section{问题}
	2D平面内的两条线段AB与CD,分析两条线段之间的关系。比如两条线段是否有交点,如果无交点,是否是平行的。
	
\section{分析}
	还是利用上次的老办法,假设交点为Q,列方程:
	\begin{eqnarray}
	A+(B-A) * t1 = Q \\
	C+(D-C) * t2 = Q
	\end{eqnarray}\\
	于是,我们有如下方程:
	\begin{equation}
	\left[^{A_x}_{Ay}\right] + \left[^{Bx-Ax}_{By-Ay}\right] * t1 = 	\left[^{Cx}_{Cy}\right] +  \left[^{Dx-Cx}_{Dy-Cy}\right] * t2
	\end{equation}
	方程$(3)$可写成:
	\begin{eqnarray}
		(A_x - C_x)(D_y - C_y)+(B_x -A_x)(D_y-C_y)*t1 = (D_X - C_x)(D_y - C_y)*t2\\
		(A_y-C_y)(D_x-C_x) +(B_y-A_y)(D_x-C_x) * t1 = (D_y-C_y)(D_x-C_x) * t2
	\end{eqnarray}
	消元t2得到:
	\begin{equation}
	t1 = \frac{(A_y-C_y)(D_x-C_x) - (A_x-C_x)(D_y-C_y)}{(B_x-A_x)(D_y-C_y)-(B_y-A_y)(D_x-C_x)}
	\end{equation}
	我不知道大家看到方程$(6)$会有什么想法,就我的直觉来说,如果等号右边的分式的分母为0,则t1不存在,也就是$$(B_x-A_x)(D_y-C_y)-(B_y-A_y)(D_x-C_x)=0$$
	如果你足够敏锐:$$\frac{B_y-A_y}{B_x-A_x}=\frac{D_y-C_y}{D_x-C_x}$$
	这让我看到斜率一样,共线或平行。\\
	我们用同样的方法得到:
	\begin{equation}
	t2 = \frac{(B_y-A_y)(C_x-A_x) - (B_x-A_x)(C_y-A_y)}{(B_x-A_x)(D_y-C_y)-(B_y-A_y)(D_x-C_x)}
	\end{equation}
	再一次印证了平行的条件的猜测。
	那么问题来了,相交的条件又是什么呢?首先:
	$$(B_x-A_x)(D_y-C_y)-(B_y-A_y)(D_x-C_x)\neq0$$
	然后,我们的t1和t2必须在$\left[0,1\right]$之间,既:
	\begin{eqnarray}
	0<=t1<=1\\
	0<=t2<=1
	\end{eqnarray}
	看起来我们算是已经解决了问题,不过得多说几句,如果我们要把这套计算过程用计算机程序来实现,真的得注意一些事情。假定$a=(A_y-C_y)(D_x-C_x) - (A_x-C_x)(D_y-C_y), b=(B_y-A_y)(C_x-A_x) - (B_x-A_x)(C_y-A_y) , c=(B_x-A_x)(D_y-C_y)-(B_y-A_y)(D_x-C_x)$。也就是$t1=\frac{a}{c}$,$t2=\frac{b}{c}$。\\
	得考虑下线段$AB$与$CD$共线时的特殊情形了,也就是$c=0$这种情况下:
	\begin{itemize}
		\item AB与CD重叠,存在一个或者多个交点
		\item AB与CD不重叠,无交点
	\end{itemize}
	先讨论第一种情况,如果AB与CD共线,我们得出:
	\begin{eqnarray}
		\frac{B_y-A_y}{B_x-A_x} = \frac{A_y-C_y}{A_x-C_x}\\
		\frac{A_y-C_y}{A_x-C_x} = \frac{D_y-C_y}{D_x-C_x}
	\end{eqnarray}
	根据对角线法则,$(10)$和$(11)$与$a$和$b$是密切相关的。另一方面,$c=0$,就几何意义上来讲,如果$a=0,c=0$,一定有$b=0$。你能看出来吗?接下来的工作是确定交点到底有多少个了,可能是一个,也可能是线段的一部分。首先我们得对点$A,B,C,D$以x坐标为参考做由左向右,由上到下的排序,得到经过排序后的线段$AB$,$CD$,要知道,排序很重要!接下来再做一次以一维相交性检测,通过少量的计算存在以下情况:
	\begin{itemize}
		\item ALL :线段CD在A点左侧
		\item BRR :线段CD在B点右侧
		\item LAR :线段CD在A点左右两侧
		\item LBR :线段CD在B点左右两侧
		\item LA :线段D点与A点重合
		\item BR :线段C点与B点重合
	\end{itemize}
	我们可以确定的是:当$a=0,b=0,c=0$时,AB与CD也是相交的。
	\section{结论:}
	\begin{itemize}
		\item 当$(B_x-A_x)(D_y-C_y)-(B_y-A_y)(D_x-C_x)\neq0$时,一定有一个交点Q;
		\item 当$(B_x-A_x)(D_y-C_y)-(B_y-A_y)(D_x-C_x)=0$时,$AB$与$CD$平行或共线,特殊的,当且仅当$a=0,b=0,c=0$时,AB与CD共线且至少有一点重合。
	\end{itemize}
\end{document}